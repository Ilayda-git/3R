\documentclass{article}

% Language setting
% Replace `english' with e.g. `spanish' to change the document language
\usepackage[english]{babel}

% Set page size and margins
% Replace `letterpaper' with `a4paper' for UK/EU standard size
\usepackage[letterpaper,top=2cm,bottom=2cm,left=3cm,right=3cm,marginparwidth=1.75cm]{geometry}

% Useful packages
\usepackage{amsmath}
\usepackage{graphicx}
\usepackage[colorlinks=true, allcolors=blue]{hyperref}

\title{Mobilité professionnel et salaire : différence entre hommes et femmes}
\author{Ilayda Yilmaz}

\begin{document}
\maketitle

\section{Introduction}

L’article de Véronique Simonnet, publié en 1996, étudie la relation entre mobilité professionnelle et salaire, en mettant l’accent sur les différences entre hommes et femmes. \\
Dans le contexte du marché du travail français du début des années 1990, la mobilité professionnelle est souvent envisagée comme un moyen d’améliorer sa trajectoire salariale.\\
Toutefois, l’auteure s’interroge sur l’égalité réelle de ce mécanisme entre les sexes : les femmes retirent-elles les mêmes bénéfices salariaux de leur mobilité que les hommes, ou bien subsistent-il des écarts liés à des mécanismes discriminatoires et à la ségrégation professionnelle ?

\section{Etudes}

\subsection{Problématique}

L’approche théorique s’appuie sur la théorie du capital humain (Becker 1975), qui assimile les connaissances et compétences acquises par un individu à un bien de production.\\
Ces investissements en éducation et en expérience augmentent la productivité et le salaire. La relation entre capital humain et rémunération a été formalisée par Mincer sous la forme suivante :

\vspace{5pt}

\textcolor{white}{12345678912345678912345}
$Ln E{it} = Ln E_{i0} + rsS_i + reE_{xi} + u_{_it} $  
\textcolor{white}{123}  où :

\vspace{5pt}

 - $E_{it}$  représente le salaire potentiel de l’individu, \\
 \textcolor{white}{123}- $i$ la période ,\\
\textcolor{white}{123}- $E_{i0}$  la rémunération associée au capital humain inné, \\
 \textcolor{white}{123}- $S_i$  le nombre d’années d’études, \\
 \textcolor{white}{123}- $E_{xit}$  l’expérience professionnelle acquise,\\
 \textcolor{white}{123}- $u_{it}$  le terme d’erreur, \\
 \textcolor{white}{123}- $Rs$ et $re$ correspondent respectivement aux rendements moyens d’un année d’études et d’une  \textcolor{white}{12345}année d’expérience professionnelle.\\

\vspace{5pt}
 
Cependant, Simonnet distingue deux formes d’accumulation du capital humain : d’une part, le capital humain général, acquis par l’expérience professionnelle et valorisante sur l’ensemble du marché du travail (marché externe) ; d’autre part, le capital humain spécifique, qui correspond aux compétences et savoir-faire développés dans une entreprise donnée et valorisantes uniquement en interne. \\
Pour prendre en compte cette dualité, l’équation de gains est enrichie par l’introduction de l’ancienneté dans l’entreprise, notée Anci . La fonction de gains devient alors :

\vspace{5pt}

\textcolor{white}{12345678912345678912345}
$Ln E_{it} = Ln E_{i0} + rsS_i + reE_{xi} + raANC_i + u_{it} $
\textcolor{white}{123}  où :

\vspace{5pt}

 - ra mesure la rentabilité de l’ancienneté, c’est-à-dire la contribution spécifique du marché interne \textcolor{white}{1234} à l’évolution salariale.

\vspace{5pt}
 
Ce cadre permet d’identifier empiriquement la rentabilité de l’expérience externe et celle de l’ancienneté interne. En outre, Simonnet introduit dans ses estimations la variable de mobilité professionnelle, en distinguant la mobilité interne (changement de poste au sein de la même entreprise) de la mobilité externe (changement d’entreprise). \\
Les équations sont ensuite estimées séparément pour les hommes et pour les femmes, de manière à comparer l’impact différencié de ces différentes formes de capital humain et de mobilité sur les salaires selon le sexe.

\subsection{Données \& Panel}

L’étude utilise les données de l’enquête Emploi de l’INSEE, portant sur un échantillon représentatif des salariés français. Ne sont retenus que les individus en emploi, ce qui évite le biais lié à la sélection dans le chômage. La mobilité est décomposée en plusieurs types : la mobilité interne (au sein de la même entreprise), la mobilité externe (changement d’entreprise) et la mobilité sectorielle.\\
Cette finesse permet d’observer si tous les types de mobilité sont valorisés de la même façon sur le plan salarial.

\subsection{Résultats}

Les estimations montrent que la mobilité a un effet positif sur le salaire, mais cet effet est beaucoup plus marqué pour les hommes que pour les femmes. Les hommes qui changent d’emploi perçoivent en moyenne une augmentation salariale significative, alors que pour les femmes, l’impact est plus faible, voire parfois nul selon le type de mobilité considéré.\\
Concrètement, le coefficient α4h  est positif et significatif, ce qui confirme que la mobilité constitue une stratégie efficace pour améliorer les salaires masculins. En revanche, α4f  apparaît beaucoup plus faible, ce qui signifie que, toutes choses égales par ailleurs, les femmes ne tirent pas les mêmes bénéfices de leur mobilité.

\section{Conclusion}

L’article de Simonnet montre que la mobilité externe est clairement rémunératrice pour les hommes, elle ne procure pas les mêmes avantages aux femmes, dont les salaires progressent beaucoup plus faiblement. \\
La mobilité apparaît donc comme un levier asymétrique : elle amplifie les gains de carrière des hommes, mais reste limitée pour les femmes, révélant que les discriminations salariales ne tiennent pas seulement aux caractéristiques individuelles, mais à des structures profondes du marché du travail.

\section{Annexe}

\begin{center}
  \setlength{\fboxsep}{0.1cm} 
  \setlength{\fboxrule}{1pt}
  \fcolorbox{col}{white}{
    \includegraphics[width=0.5\linewidth]{Images/Echantillon.png}}\\
    \vspace{0.1cm}
    \caption{Tableau d'échantillon}
\end{center}

\end{document}